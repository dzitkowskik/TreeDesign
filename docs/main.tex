\title{
02257 Applied functional programming \\
Project 2\\
Drawing Trees in F\#
}
\author{
        Karol Dzitkowski - s142246,
        Mads Egedal Kirchhoff - s093027
}
\date{\today}

\documentclass[12pt]{article}
\usepackage[utf8]{inputenc}

\begin{document}
\maketitle
\setlength{\parindent}{0pt}
With the signatures the group membersaccount for:
\begin{enumerate}
\item The extension from the minimal solution is made by the group
members only
\item No part of their solution is distributed to other groups
\item The group members have contributed equally to the solution
\end{enumerate}
Group members:
\begin{itemize}
\item Karol Dzitkowski - s142246
\item Mads Egedal Kirchhoff - s093027
\end{itemize}
Signatures:
\clearpage

\section{Status of the solution}
We succeeded in translating the solution to $F\#$ language and it is working for all trees translated 
to general trees from previous project (11 smaller and bigger trees). We also implemented a function
translating Abstract Syntax Trees to general trees. We wrote a function to create a PostScript file
containing a picture of a general tree and an algorithm is working for all the trees we tested. 
Analysing the efficiency of our solution we conclude that our results coincide with the expected, presented
in the article we got. We messured an execution time of 3 major parts of our project:
\begin{enumerate}
\item Translation (Small tree 0.023 ms, big tree 0.036 ms)
\item Generating design tree (Small tree 0.271 ms , big tree 0.402 ms)
\item Generating Post Script from design tree (Small tree 0.710 ms, big tree 0.933 ms) 
\end{enumerate}

\section{Extensions}
We implemented two versions of Generation Post Script from design tree function. First of them is 
using StringBuilder for all string concatenations, while another is using String.concat method. The
results of the efficiency comparison are of course in favor of the StringBuilder method. For small trees
are usually not significant (for example SB time 0.7815 ms and String.concat time 0.8947). However
sometimes the difference is greater like: 0.8562 ms and 9.3201 ms respectively. For big trees the time
efficiency difference was always significantly better for SB (0.6496 ms to 5.2416 ms).

\section{Reflections}
Woring on the trees in Functional Programming language is very convenient. We learned that using
StringBuilder can have a significant influence on efficiency and sometimes can even be the most
comfortable way of concatenating many strings. It also helps with line endings since we can use an
 $AppendLine$ function. An natural extension should be also making a size of the Post Script page
 depending on the size of the tree (its output maximal width and height). 

\end{document}